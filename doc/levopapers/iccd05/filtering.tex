\vspace{-0.25in}
\subsection{Result Forwarding Buses and Operand Filtering}
\vspace{-0.15in}
%
The basic requirement of the interconnection fabric is that it must be able
to transport operand results from any instruction to those
instructions with higher valued time tags.  This corresponds
to the forwarding of operands into future program ordered time.
There are several choices for a suitable interconnection fabric.
A simple interconnection fabric could consist of one or more shared
buses that simply interconnect all issue slots.
Although appropriate for smaller sized microarchitectures,
this arrangement does not scale as well as some other alternatives.
A more appropriate interconnection fabric that would allow
for the physical scaling of the microarchitecture may be one in which
forwarding buses are segmented with active repeaters between
the stages.
This arrangement exploits the fact that register lifetimes
are fairly short~\cite{Franklin92,Sohi95}.
Registers being forwarded to instructions lying close
in future program-ordered time will get their operands quickly
while those instructions lying beyond the repeater units
will incur additional delays.
In addition to allowing for physical scaling, it also offers
the opportunity for filtering out some operands that do
not need to be forwarded beyond a certain point.
Another possibility for the operand forwarding fabric is
to have separate buses for each type of operand.
This sort of arrangement could be used to tailor the available
bandwidth provided for each type of operand.  It could also
allow for a different interconnection network to be used
for each type of operand also.  We have explored several of
these possible variations already.

The opportunity to provide active repeaters between forwarding bus
segments also opens up a range of new microarchitectural
ideas not easily accomplished without the use of time tagging.
Different types of active repeaters can be optionally used.
Further, different types of repeaters can be used for
the different types of operands.
Some can just provide a store-and-forward function while
another variation could also store recently forwarded operands
and filter out (not forward) newly snooped operands that
have already been forwarded previously. 
The latter type of forwarding repeater unit is termed a
\textit{filter unit}.
This feature can be used to reduce the operand traffic
on the forwarding fabric and thus reduce implementation costs.
For example, for memory operands, a cache (call it a \textit{L0 cache})
can be situated inside a \textit{memory filter unit} (MFU)
and can hold recently requested memory operands.
This can reduce the demand on the L1 data cache and the rest
of the memory hierarchy by satisfying some percent of memory
operand loads from these caches alone.
Register filter units (RFUs) are also possible and can reduce
register transfer traffic similarly to the MFUs.
%
