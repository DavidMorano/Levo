%
%\vspace{-0.25in}
\subsection{Varying the number of issue stations}
%\vspace{-0.15in}
%
One of the more obvious aspects of our microarchitecture
that contributes to the amount of ILP
that can be extracted from the programs is the number of Issue Stations
implemented.
Here we simulate five machines with
varying numbers of Issue Stations.
We present two data sets here.  
We began by configuring a machine with  
8 integer ALUs and then extended this to 16 ALUs.
A machine  
with 128 Issue Stations and eight integer ALU FUs resembles
a typical machine in development today.
The taget design point is centered around a next generation
machine that has an issue width of 16 rather than 8.

The IPC results for all 
benchmarks along with the harmonic mean 
is shown for this first data set (8 integer ALU function units)
in Table \ref{tab:iwsize8ipc}.
The average percent of instruction executions that are 
actually re-executions ranged from about 1.2\% for the configuration
with 16 Issue Stations to about 2.2\% for 256 Issue Stations.
Although only 8 integer FUs are employed, IPCs for a program
can be greater than 8 since other instructions are executed
either within the issue station itself or by other FU types.
Notice that the highest IPC performance is not achieved with
the largest number of Issue Stations but rather with the configuration
with 64 Issue Stations (having a harmonic mean of 8.0).  
The IPC starts to decrease due to
the destructive contention for function units (actually the integer
ALU units).

%
\begin{table}[p]
\begin{center}
\caption{{\em IPC performance for varying numbers of
Issue Stations.}
All machines had 8 integer ALU FUs with varying ISs from 16 to 256.}
\label{tab:iwsize8ipc}
\vspace{+0.1in}
\begin{tabular}{|l||r|r|r|r|r|}
\hline 
{ISs}& 16 & 32 & 64 & 128 & 256 \\
\hline

\hline
bzip2&
5.1 & 7.7 & 7.4 & 5.8 & 3.6 \\

\hline
crafty&
4.9 & 7.7 & 8.1 & 7.3 & 6.1 \\

\hline
eon&
4.2 & 6.6 & 8.3 & 8.0 & 6.0 \\

\hline
gcc&
5.2 & 8.0 & 8.0 & 6.8 & 6.7 \\

\hline
gzip&
5.2 & 8.0 & 8.1 & 7.6 & 5.6 \\

\hline
parser&
5.1 & 7.6 & 8.0 & 7.3 & 7.6 \\

\hline
perlbmk&
5.1 & 8.1 & 7.2 & 5.9 & 3.6 \\

\hline
twolf&
4.8 & 7.5 & 7.7 & 7.3 & 5.2 \\

\hline
vortex&
5.3 & 8.9 & 9.0 & 8.1 & 5.5 \\

\hline
vpr&
4.7 & 7.5 & 8.7 & 8.4 & 6.5 \\

\hline
H-MEAN&
4.9 & 7.7 & 8.0 & 7.1 & 5.3 \\

\hline
\end{tabular}
\end{center}
\end{table}
%
%
