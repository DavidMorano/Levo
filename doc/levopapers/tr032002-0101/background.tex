%
\section{Background on Multipath Execution}
%
There have been several attempts at substantially increasing
program IPC through the exploitation of ILP.
The Warp Engine~\cite{Cle95}
used time-tags in order to manage large amounts of speculative execution,
but their management of them is different than ours.
The Multiscalar processor architecture \cite{Soh95}
is another attempt at
realizing substantial IPC seedups over convention superscalar
processors.  However, our approach is quite different than theirs
and their approach relies on compiler participation where we do not.
A notable attempt at realizing high IPC was done by
Lipasti and Shen on their Superspeculative
architecture~\cite{Lip97}.  They achieved an IPC of
about 7 with realistic hardware assumptions.
The Ultrascalar machine~\cite{Hen00}
achieves {\em asymptotic} scalability,
but only realizes a small amount of IPC due to its 
conservative execution model.
Nagarajan et al proposed a {\em Grid Architecture} of ALUs
connected by a operand network~\cite{Nag01}.  
This has some similarities to our work.
They have shown a rather impressive IPC of 11 on integer benchmarks.
However, unlike our work, their microarchitecture
relies on the coordinated use of their compiler to obtain the high IPCs.

Early work on multipath execution was
dominated by IBM in the
late 1970s and 1980s \cite{Conners79}.
The earliest attempts at multipath
execution started with the ability to prefetch down both
outcomes of a conditional branch.  This became more aggressive
to the point of actually executing down both outcomes of
a conditional branch.  This has
been explored in work such as that by
Wang \cite{Wang90}.  
More aggressive research by Uht and
Sindagi \cite{Uht95} explored the intersection of both
multipath execution and future large-scale microarchitectures
capable of possibly hundreds of instructions being executed simultaneously.
They also addressed the general question of speculatively executing
more than two paths simultaneously.
Work on dual path execution (only two speculative paths) has
been done by Heil and Smith~\cite{Heil96}.
Klauser et al explored multiplath execution (including more than two
speculative paths)
on the PolyPath microarchitecture.

Exploring multipath execution starting from a simultaneous multithreading (SMT)
microarchitecture has been done by
Wallace et al \cite{Wallace98}.  
Ahuja et al \cite{Ahuja98} explore some limits for speedups from
multipath execution but their work is still largely restricted to more
conventional (modest sized) microarchitectures with less than approximately
128
speculative instructions in flight.  Our present work explores the use
of multipath execution on a significantly larger microarchitecture than
that of Ahuja or the other past work with the exception of that by
Uht and 
Sindagi \cite{Uht95}.
