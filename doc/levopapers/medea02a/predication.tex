
Also employed within the execution window is a scheme to
dynamically predicate, at execution time, 
all instructions that have been loaded
into active stations.  
This predication scheme essentially provides for each loaded instruction
an \textit{execution predicate}.  These execution predicates
are just a single bit (like with explicit architectural predication)
but are maintained and calculated within the microarchitecture
itself and are thus not visible at the ISA level of abstraction.
When the value of the execution predicate is \textit{TRUE}, the
instruction is enabled to execute and possibly become a part
of the final committed state of the executing program.
When the execution predicate is \textit{FALSE}, the
corresponding instruction is disabled (or not enabled) to
become a part of the committed state of the program.
Switches between being enabled and disabled can and do
occur as the speculative outcomes of previous (in program order)
conditional branches change.
These predicates are handled very similarly (possibly as expected)
to register operands that flow from previous program ordered
instructions to later program ordered instructions.
These predicate values and are rightfully control-flow operands
just as the machine registers form data-flow operands.
All instructions acquire input source predicate values (possibly
more than just one) and calculate their execution predicate
from the possible inputs.
Conditional branches are the primary instructions that
cause predicate value changes but our present scheme (and likely
most any scheme -- we have explored to some extent three schemes so far) 
is more complex than that just having conditional branch instructions
manipulate the predicate operand value flow.
In addition to conditional branches,
the first control-independent instructions after their
corresponding branches also participate in the calculation
of predicate values.  
Of course, any execution-time predication scheme needs to
handle any combination of control-flow that may possibly occur
and this is not an entirely trivial matter.
A full discussion of the execution-time predication scheme
is beyond the scope of this present paper.

