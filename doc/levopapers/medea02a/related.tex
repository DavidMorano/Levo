%
\section{Related Work}
%
Probably the most successful high-IPC machine to date is
Lipasti and Shen's Superspeculative
architecture~\cite{Lip97}, achieving an IPC of
about 7 with realistic hardware assumptions.
The Ultrascalar machine~\cite{Hen00}
achieves {\em asymptotic} scalability,
but only realizes a small amount of IPC,
due to its conservative execution model.
The Warp Engine~\cite{Cle95}
uses time tags, like Levo, for a large amount of speculation;
however their realization of time
tags is cumbersome, utilizing floating point
numbers and machine wide parameter updating.

Nagarajan et al. have proposed a {\em Grid Architecture} that
builds an array of ALUs, each with limited control, connected
by a operand network~\cite{Nag01}.  Their system achieves an IPC of 11 on
SPEC2000 and Mediabench benchmarks.  While this architecture
presents many novel ideas in attempt to reap high IPC, it
differs greatly in its interconnect strategy and register design.
They also rely on a compiler to obtain this level of IPC, whereas
the microarchitecture that we have presented does not.
