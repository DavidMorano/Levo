%
\subsubsection{Register Filtering Unit}
%
This machine component is primarily used to forward operands
from one physical forwarding bus segment to another.
This component is used to handle register operands only.
Other types of operands (like predicate and memory) are 
handled by components
suited to their operand type.
This unit can properly be termed a 
a \textit{forwarding unit}
but its function
is slightly enhanced over just a basic forwarding function.
In addition to just repeating a received operand onto an outbound
bus segment, this unit also serves to reduce the number of outbound
forwarding operations through a filtering process (hence its name).
Although the details of the filtering process is not discussed
in detail, basically the unit repeats received operands
from program ordered past to bus segments that server the
ASes in program ordered future.
In Figure \ref{fig:window}, program ordered past is on the
top side of the unit and program ordered future is on the bottom
side.
Essentially, copies of register operands are saved 
within
the unit itself and when a new operand is received, a search is
made to see if the last register with the same address
(the architected register name) that was forwarded has the same value
as the current one.  If the values are different, the new operand
is stored and forwarded.  If the values are the same, the present
operand is not forwarded since it would not result in changing
any data dependencies in the program.  
This process effectively filters out unnecessary forwarding
operations that do not impinge on proper program register-data
dependencies.
Thus, this mechanism serves to
reduce the required forwarding bus bandwidth.
%
\subsubsection{Predicate Forwarding Unit}
%
This unit is very similar to the register filtering unit but
is presently simpler in this design and function.
This unit basically just consists of a FIFO to buffer up
received operands until they can be repeated on the outbound bus segment.
Like the register filtering unit, operands are received 
from program ordered past and repeated on the outbound bus
segment that serves ASes in program ordered future.
The outbound bus generally has several bus masters, all contending
for a time slot to writing operands, so generally operands
may have to wait one or more clocks before they can be repeated.
Future work may include enhancements to this unit but
it is currently believed that the numbers of predicate operand
forwarding operations that occur is not high enough to be
a critical bottleneck to performance.
%
\subsubsection{Memory Filtering Unit}
%
This unit, like the other forwarding units, serves to repeat
received operands from one bus segment to another.
However, unlike the other forwarding units, this unit is somewhat
more complicated due to the fact that the address space for
memory is so large (at least $ 2 ^ { 32 } $).
This unit is also frequently termed an \textit{L0 data cache}
since it indeed effectively has a cache of speculative
memory values stored within it.
We will subsequently be using the terms \textit{L0 data cache}
and \textit{Memory Filtering Unit} somewhat interchangeably
depending on the context of its discussion.
Like a register filtering unit, this unit attempts to
filter out unnecessary forwarding operations in order to
reduce the bandwidth requirements of the forwarding interconnection
fabric.

Due to topological differences in the memory forwarding interconnection
fabric from either the register or predicate forwarding interconnection
fabric, this unit has slightly differing responsibilities depending
on where it is located in the fabric.  Further, unlike the other
forwarding units, this unit repeats
operands bidirectionally.
It receives operands from all of its connecting buses
and repeats them on the opposite bus ports.
This is more like a flooding algorithm than the loop-around
schemes current used for registers and predicates.
Operands eventually reach the hub of the star topology of the
interconnection fabric and then get forwarded out to the
other leafs of the fabric.  In this way, all ASes get to
snoop on any memory value that they may be operating on.
%
