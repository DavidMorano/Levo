paragraph{Examples : }

Some simple examples of some code snippets will be 
given in this section in order to illustrate the snooping/snarfing
operation within an active station.

A snippet of code that we will consider for an example
follows.

\begin{verbatim}

00	r9 <- r0 op r1
10	r3 <- r5 op r6
20	r3 <- r7 op r8
30	r2 <- r3 op r8

\end{verbatim}

In this example we will look at how instruction number 30 gets its
correct value for one of its sources ; namely, register
{\it r3}.
Instruction number 00 does not produce
any outputs used in the next few instructions that
we will consider so there is no data dependency and its output 
is never snooped by them.  The execution of instruction 00 therefor
does not further impact the correct operation of instruction 30.
We will assume that each of these
instructions are loaded into active stations with the same row designation
as the instruction number.  
At load time, instruction 30 will
load register
{\it r3}
from the ISA register file as an initial value guess.  
This is, of course,
probably not the correct value that instruction 30 should be using but 
we use it anyway as a predicted value.
In this example, instructions numbered 10 and 20 both generate
an output to register 
{\it r3}.
so instruction 30 will snoop
for this register output changing.  
It will watch for the address
of that register (the number {\it 3} in this example) to
appear on one of the 32 data output forwarding buses.
Initially when the instructions are loaded,
they may all execute immediately.  Let's assume that this happens
and all instructions execute in the same clock immediately upon being
loaded.
Instruction 3 will now have a result based on what register
{\it r3}
was at load time but a new output for this
register has just been produced by instructions 1 and 2.
Instructions 1 and 2 will both broadcast forward their new output
values for
{\it r3}
on separate forwarding buses.
Instruction
30 is snooping for a new update
to register
{\it r3}
on all forwarding buses.  Updates coming from both instructions
1 and 2 will be considered for snarfing by instruction 3 since
its forwarding bus time tag mask is initially set to all bit set.
Instruction 3
will see that the output from instruction 20 is later
in time than that from instruction 10, so it will snarf the value
and update its forwarding bus time tag mask by clearing all bits
to the right (earlier in time) than the bit corresponding to
instruction 20 above.  No register updates from instruction 10
will ever again now be snarfed by instruction 30.  New updates
from instruction 20 will still be considered though since its
bit in the forwarding bus time tag mask is still set.
Execution of this example is illustrated in
Figure~\ref{ex1}
where instruction executions are shown in time where
there is an '{\it X}' at an intersection of an instruction
and a time clock value.

Another example, still referencing the code snippet above,
is that upon the instructions all being loaded in the same
clock period, only instruction 10 can execute immediately.
Again, we focus of what instruction 30 does to get a correct
value for register
{\it r3}.
After being executed, instruction 10 will broadcast 
its updated value for register
{\it r3}
on a forwarding bus.
This value will be snarfed by instruction 30 since all of its
time tag mask bits are still clear.  Upon snarfing the updated
value from instruction 10, all time tag bits to the right of
it will be cleared to zero.  In our current example, there
were no instructions earlier than instruction 10 within the instruction
window which generated an output to register
{\it r3}
but the bits are cleared just the same.
Instruction 30 will now
use the updated register value the next time it gets to execute.
Also, if it had already executed, the act of snarfing a value will
enable it to be executed again.  Of course, the newly broadcast
value would not have been snarfed it its value did not change.
Finally, instruction 20 gets a chance to execute and afterwards it
broadcasts its output value on a forwarding broadcast bus.
Since it is a later in time instruction than instruction 10,
the bit in the time tag mask of instruction 30 which corresponds
to instruction 20 will still be set indicating that a snarf
is still possible from that instruction.
Execution of this example is illustrated in
Figure~\ref{ex2}.




