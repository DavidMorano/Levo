
paragraph{Sharing Groups and Result Forwarding: }
Active stations within the instruction window are grouped together in
order to allow for sharing of expensive execution resources.  Expensive
execution resources can range from an entire processing element that
can execute any instruction to specialized functional units.
Implementations can also include the situation of having just one
active station in a sharing group.  Figure ?? above contained four
sharing groups within the instruction window shown there.

Execution output results (intended to be located in the ISA architected
registers) from sharing groups must be forwarded to those active
stations located forward in program execution order (having later
valued time tags).  This is accomplished with result forwarding
busses.  These forwarding buses are shown in Figure ??.  Results are
not necessarily immediately forwarded to all later active stations.
Because instruction output results in the architected registers only
need to be forwarded for the lifespan of the ISA register in the
program, results are only first forwarded a number of active stations
that roughly matches a typical register lifetime.  This forwarding
distance is termed the {\it forwarding span}.  Each active station
snoops all forwarding buses to get inputs needed for its execution.
Each sharing group originates a forwarding bus that covers the
implementation's designated forwarding span.  For example, if an
instruction window consists of 256 active stations and these are
further divided into sharing groups containing eight active stations
from the ML column (and eight from a DEE column) then there would be 32
sharing groups each originating a forwarding bus.  This means that at
any active station there would be 32 forwarding buses that would need
to be snooped for source inputs.

In those situations where an output result from an instruction is
needed in instructions located beyond the forwarding span of its
forwarding bus, there exists a register at the end of the bus (located
in the last sharing group that is snooping that bus).  This register is
termed the {\it forwarding register}.  This register then contends
with the forwarding requests originating in its sharing group to
forward its result on that sharing group's forwarding bus.  The process
will result in output values being forwarded for the next forwarding
span number of active stations.  This process can be repeated as needed
to continue forwarding results until another instruction has that ISA
register designated as its output.  This forwarding process is stopped
when any active stations within the forwarding span of a result has the
same ISA register as an output.  The later instruction originating a
new value for that ISA register is now responsible for forwarding its
output value in the same manner.

Note that when instruction output results need to be forwarded to
stations beyond the implementation forwarding span there is at least
one clock cycle delay in the forwarding process due to the presence of
the forwarding register.  This register minimally needed because of
possible contention for the forwarding bus of the sharing group the
forwarding register is located in.  Note that more than one clock may
be incurred if the sharing group that is performing the forward also
needs to forward one of its results in the same clock cycle.  Delaying
a forward from a previous sharing group will not typically be a
performance problem since the need for a result created a long time in
the past is not as urgent as needing a result that was generated in the
more recent past of the program instruction stream.



