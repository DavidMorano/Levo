%
\subsubsection{I-fetch Unit}
%
The i-fetch unit has some additional responsibilities beyond just
the fetching of instructions from the most predicted execution path
of the program.  
Some of the responsibilities of our i-fetch unit are :
%
\begin{itemize}
\item{fetching of instructions along predicted program paths}
\item{decoding of instructions}
\item{making initial predictions for conditional branch outcomes}
\item{staging of instructions for loading into the execution window}
\item{storage of instruction traces for possible loading into the
execution window}
\end{itemize}   
%
With the exception of the storage of instruction traces,
these functions are rather conventional.
The i-fetch unit first fetches instructions from i-cache
along one or more predicted program paths.
In our microarchitecture, instructions are immediately
decoded after being fetched.
All further handling of the instructions is done in their decoded
form.
The primary function is to stage decoded instructions 
into a \textit{load buffer}
so that they are available to be shifted into the execution
window.  This load buffer is organized so that
a large number of instructions can be broadside loaded into the
execution window in a single clock.
The multiple buses going from the i-fetch unit to the
execution window in Figure \ref{fig:high} is meant to
reflect this operation.  Six buses are shown in that figure
and the significance of that number will shown in the
section that discusses the execution window.
The number of instructions that can be loaded simultaneously
in a single clock will be later referred to as 
the \textit{column height} of the machine.  This meaning
of this will become
evident in the section discussing the execution window.

In addition to staging one set of instructions to be
next loaded into the execution window, the load buffer
may optionally have several sets of instructions prepared
for a possible load operation.
The relationship between the multiple sets of staged instructions
might be one of a FIFO arrangement or something more complicated.
A FIFO arrangement only allows the set of instructions closest
to the execution window to be loaded next and often requires a
complete flush of the entire FIFO if program control flow deviates
from what was predicted by the i-fetch unit (a partial flush
is possible but it would always be from the head of the FIFO).  
A more complicated
structure is also possible where the next set of instructions
to be loaded can come from any one of several sets that
have been prepared and are waiting in the load buffer.  
This latter arrangement may offer some additional flexibility
due to control flow requirements by the execution window.
The goal is to try and always satisfy the need for the
next proper set of instructions by the execution window.
This load buffer is similar in
many respects to an instruction prefetch buffer of a sort,
except that all instructions at this stage in our machine
are already fully decoded.

The i-fetch unit may also optionally have a cache of
instruction traces.  We have not explored this option 
in detail as yet
but it is a rather obvious one based on similar work.
In our case, traces of instructions would correspond closely
with the sets of instructions that need to be staged
in the load buffer (column height worth of instructions).
These stored instruction traces would allow for the quick
preparation of a new set of instructions to be loaded
into the execution window when a control flow misprediction
requiring a major machine flush might occur.
Obviously, the flushing of large numbers of instructions
anywhere in the machine is undesirable but keeping the
execution window satisfied with new instructions is also
a primary concern.  Our machine performance so far has
not prompted us to more aggressively explore the management
of the caching of instruction traces in our i-fetch unit
but this is certainly something that may be useful as
we explore larger machines than we are at the moment.
