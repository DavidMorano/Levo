Title: Realizing High IPC Through a Scalable Memory-Latency Tolerant
Multipath Microarchitecture

Authors:

D. Morano, A. Khalafi, D.R. Kaeli
Northeastern University
akhalafi,dmorano,kaeli@ece.neu.edu
   
A.K. Uht 
University of Rhode Island
uht@ele.uri.edu

Contact author: David Kaeli (kaeli@ece.neu.edu)

Abstract:

A microarchitecture is described that achieves high execution
performance through high IPC on conventional single-threaded program
codes without compiler assistance.  To obtain high IPC 
for inherently sequential (e.g., SPECint2000 programs),
a large number of instructions must be in flight simulateously.   
However, several problems are associated with
such microarchitectures, including scalability, issues related to
control flow, and memory latency.

Our design investigates how best to utilize a large mesh of processing
elements in order to execute a single-threaded program.  
We present a basic overview of our microarchitecture and discuss how it
addresses scalability as we attempt to execute many instructions in
parallel.  We also show how spawn disjoint execution 
paths to limit the
performance penalties associated with erratic control flow.  We provide
simulation results for several geometries of our microarchitecture
illustrating a range of design tradeoffs.  We particularly focus on 
the tolerance of our microarchitecture to high memory latency.
