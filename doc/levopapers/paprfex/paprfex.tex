\documentclass[10pt,dvips]{article}
\usepackage{epsfig}
\usepackage{fancyheadings}
%\usepackage{twocolumn}
\usepackage{verbatim,moreverb,doublespace}
\usepackage{rotate,lscape,dcolumn,array,rotating,latexsym}

\topmargin=-0.6in
\setlength{\oddsidemargin}{0.0 in}
\setlength{\textheight}{9.0in}
\setlength{\textwidth}{6.5in}
%\renewcommand{\thepage}{2}
\renewcommand{\baselinestretch}{1.0}

\begin{document}

\begin{landscape}
\begin{table}
\begin{tabular}{lllllll}
I Window & Iteration 1 & Iteration n & Instruction & Operands & \multicolumn{1}{l}{Inputs/Output Memory Address (if any)}
	& Outputs (Iter. 1) \\
position & exec. time  & exec. time  &             &          &
        &     \\   
\hline  
65 & 3	   & 3 NC &	addiu	& sp,sp,-80	& $<$sp=0x7fff2d00$>$	&				sp=0x7fff2cb0\\
66 & (4),5 & 4 NC &	sd	& gp,0(sp) 	& $<$0x7fff2cb0$>$	$<$gp=0x1001c3fc,sp=0x7fff2cb0$>$ &	gp=0x1001c3fc\\
67 & 5,7   & 5 NC &	addiu	& at,at,-26388	& $<$at=0x20000$>$				&	at=0x198ec\\
68 & 6,8   & 6 NC &	addu	& gp,t9,at	& $<$at=0x198ec,	t9=0x10002b10,gp=0x1001c3fc$>$ &	gp=0x1001c3fc\\
69 & 9     & 7    &	lw	& v1,-32508(gp) & $<$0x10014500$>$	$<$v1=0x0,gp=0x1001c3fc$>$	&	v1=0x100143f0\\
70 & 10	   & 8 NC &	li	& a2,8		& $<$a2=0x1001c628$>$				&	a2=0x8\\
72 & (11),12 & 9  &	lw	& a5,-32368(gp) & $<$0x1001458c$>$	$<$a5=0x5020,gp=0x1001c3fc$>$ &	a5=0x9\\
71 & 12    & 10 NC(pred) &	bltz	& a0,$<$far away$>$	& $<$a0=0x20$>$ & \\
& & & & & & \\
73 & (3),(4),5 & 4    &	sd	& s0,24(sp) 	& $<$0x7fff2cc8$>$	$<$s0=0x20,sp=0x7fff2cb0$>$	&	s0=0x20\\
74 & 3         & 3 NC &	lui	& s0,0x6	& $<$s0=0x20$>$				&	s0=0x60000\\
75 & 5,12      & 5,11 &	andi	& a7,a5,0x7	& $<$a5=0x9,a7=0xbd69$>$			&	a7=0x1\\
76 & (6),(7),8,(9),10 & 6,[7] NC &	lw	& a6,-32740(gp) & $<$0x10014418$>$	$<$a6=0x500020,gp=0x1001c3fc$>$ &	a6=0x10010000\\
77 & (9),11    & 7    &	sd	& s6,40(sp) 	& $<$0x7fff2cd8$>$	$<$s6=0x500020,sp=0x7fff2cb0$>$ &	s6=0x500020\\
78 & 10        & 8 NC &	ori	& s0,s0,0x51e1	& $<$s0=0x60000$>$				&	s0=0x651e1\\
79 & 13        & 9,12 &	sllv	& at,a0,a7	& $<$at=0x198ec,a0=0x20,a7=0x1$>$		&	at=0x40\\
80 & (14),15   & 10 NC &	lw	& a1,-32504(gp) & $<$0x10014504$>$	$<$a1=0x1,gp=0x1001c3fc$>$	&	a1=0x10014400\\
\end{tabular}

\caption{\textit{Sample code execution in Levo, from SPECint92-compress.}
Trace obtained via {\tt dbx}; execution times determined by hand. Two 8-instruction
sharing groups shown from Instruction Window column 3, each group
has one Processing Element. `Iteration n' times are
relative. `NC' - no change in
instruction output from first iteration. `NC(pred)' - branch input has changed, but
output predicate has not. `(\#)' - address calculation time. `[\#]' - no change in
output.}
\label{levoex}
\end{table}

An example of code execution in Levo is given in Table~\ref{levoex}. The code is from
column 3 of the instruction window, so code execution begins at cycle 3 (window loading
started at cycle 0). During the first iteration extra time is taken for multiple
executions of the same instruction. However, this does not happen in subsequent iterations
of the code.

Note that although the iterations take several cycles to execute, similar
execution patterns happen concurrently in Levo's other 30 sharing groups, so much ILP
is realized. This is an example of typically {\it locally serial} but {\it globally
parallel} code execution.

Also note that in later iterations (`n') much of the code results are unchanged (`NC').
Such instructions still take a cycle in order to obtain input operands over the
register bus. Clearly, if the data paths are increased to allow more input operands
to be fetched concurrently, iteration `n's would execute in much less time.
\end{landscape}
\end{document}


